\documentclass[11pt,a4paper]{report}
\usepackage[spanish,es-nodecimaldot]{babel}	% Utilizar español
\usepackage[utf8]{inputenc}					% Caracteres UTF-8
\usepackage{graphicx}						% Imagenes
\usepackage[hidelinks]{hyperref}			% Poner enlaces sin marcarlos en rojo
\usepackage{fancyhdr}						% Modificar encabezados y pies de pagina
\usepackage{float}							% Insertar figuras
\usepackage[textwidth=390pt]{geometry}		% Anchura de la pagina
\usepackage[nottoc]{tocbibind}				% Referencias (no incluir num pagina indice en Indice)
\usepackage{enumitem}						% Permitir enumerate con distintos simbolos
\usepackage[T1]{fontenc}					% Usar textsc en sections
\usepackage{amsmath}						% Símbolos matemáticos
\usepackage{listings}
\usepackage{longtable}
\usepackage{subcaption}
\usepackage{datatool}
\usepackage{filecontents}


% Comando para poner el nombre de la asignatura
\newcommand{\asignatura}{Simulación de Sistemas}
\newcommand{\autor}{Adrián Acosa Sánchez}
\newcommand{\titulo}{PRÁCTICA 3}
\newcommand{\subtitulo}{Modelos de Simulación Dinámicos y Discretos}
\newcommand{\rama}{Computación y Sistemas Inteligentes}

% Configuracion de encabezados y pies de pagina
\pagestyle{fancy}
\lhead{\autor{}}
\rhead{\asignatura{}}
\lfoot{Grado en Ingeniería Informática}
\cfoot{}
\rfoot{\thepage}
\renewcommand{\headrulewidth}{0.4pt}		% Linea cabeza de pagina
\renewcommand{\footrulewidth}{0.4pt}		% Linea pie de pagina

\usepackage{graphicx}
\begin{document}
\pagenumbering{gobble}

% Pagina de titulo
\begin{titlepage}

\begin{minipage}{\textwidth}

\centering

%\includegraphics[scale=0.5]{img/ugr.png}\\
\includegraphics[scale=0.3]{img/logo_ugr.jpg}\\[1cm]

\textsc{\Large \asignatura{}\\[0.2cm]}
\textsc{GRADO EN INGENIERÍA INFORMÁTICA}\\[1cm]

\noindent\rule[-1ex]{\textwidth}{1pt}\\[1.5ex]
\textsc{{\Huge \titulo\\[0.5ex]}}
\textsc{{\Large \subtitulo\\}}
\noindent\rule[-1ex]{\textwidth}{2pt}\\[3.5ex]

\end{minipage}

%\vspace{0.5cm}
\vspace{0.7cm}

\begin{minipage}{\textwidth}

\centering

\textbf{Autor}\\ {\autor{}}\\[2.5ex]
\textbf{Rama}\\ {\rama}\\[2.5ex]
\vspace{0.3cm}

\includegraphics[scale=0.3]{img/etsiit.jpeg}

\vspace{0.7cm}
\textsc{Escuela Técnica Superior de Ingenierías Informática y de Telecomunicación}\\
\vspace{1cm}
\textsc{Curso 2022-2023}
\end{minipage}
\end{titlepage}

\pagenumbering{arabic}
\tableofcontents
\thispagestyle{empty}				% No usar estilo en la pagina de indice

\newpage

\setlength{\parskip}{1em}

\chapter{Mi Segundo Modelo de Simulación Discreto}
\newpage
\section{Ejecución del modelo}

Antes de empezar, ejecutaré tanto el modelo con incremento fijo como con incremento variable con los mismos datos. Usaré un número fijo de clientes a atender (el que se propone en el guión) y las siguientes medidas de tiempo para las distintas ejecuciones del mismo:

\begin{itemize}
	\item{tlleg = 0.15, tserv = 0.1 (horas)}
	\item{tlleg = 4.5, tserv = 3 (medias horas)}
	\item{tlleg = 6.75, tserv = 4.5 (cuartos de horas)}
	\item{tlleg = 9, tserv = 6 (minutos)}
	\item{tlleg = 540, tserv = 360 (segundos)}
	\item{tlleg = 5400, tserv = 3600 (décimas de segundo)}
\end{itemize}

Una vez establecidos los datos para las ejecuciones, procedemos a ver los resultados obtenidos para el modelo con incremento fijo del tiempo.

\subsection{Incremento fijo del tiempo}

Los datos obtenidos en el caso del incremento fijo del tiempo son los siguientes:


\end{document}

